\section{Analysis of DBMS design}
In every single relation in our ERD, every single attribute is functionally dependent on the "id" attribute, as it alone ensures that there can be no absolute dublicates. Simultaneously we need to control, that when not considering the id attribute's existance, that there can be no two entries, where the only attribute taking them apart from each other is the id attribute.
Therefore we will go through and identify all possible candidate keys for all our relations, making sure that there is not two candidate keys, involving the same attribute more than once (Again: Excluding id from this candidate key search). Associated assumptions that would have to be made for a candidate key to be valid, will be mentioned in a column of its own.
See the table below for the result of this relation search exhaustion.\\

<<TABLE>>\\

It doesn't make sense to go through any of the other tables, as pretty much all of them do not work independently, and are thus bound by their stored foreign keys being part of their primary key, making it trivial to look for possible candidate keys that could form functional dependencies.\\
In the case the above table doesn't implicitly clarified that our database design is in Boyce Codd normal form we've listed a few discussion points.\\ 