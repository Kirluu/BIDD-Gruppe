\textbf{Data cleaning:} \\
During data cleaning, we found that there was indeed references pointing from the user table toward zipcode tuples that did not exist. We thus deleted these users, to avoid any errors that may occur in the future. \\
After this, we moved on to look at the rating table, making sure that the ratings of the users we just deleted were also cleaned. Finally, we went ahead and checked the moviegenre table for invalid references, however there appeared to be none. \\
Following this, we went ahead and applied proper structure to the database, by adding constraints that were otherwise not present - this was not mentioned in the assignment, but made our work a tad easier when executing various queries. \\
On the subject of the age data, we initially took the minimum age, and added it to a random value which was multiplied with the interval's length, to give some value in between. It appeared, however, that we had very few cases of the minimum age within each age group when we computed these ages. This was a problem, so we altered our update queries to 'FLOOR' the result, so that a RAND() computed value of 0.999, multiplied by 10, would, instead of becoming 10, become 9. This also meant that we could write the age intevals as their minimum up to their maximum allowed value minus one, since we can count on the flooring functionality to give us the results we want, as long as we define our numbers correctly. \\
On a side-note, we decided to make the first (youngest users) interval go from the age of 10 to 18, instead of 0-18, to make our estimates of age more realistic. \\
See the relevant SQL queries for data cleaning below: \\

%Should be SQL listing here:
Deletion-cleaning:\\
-- user
DELETE FROM user
WHERE NOT EXISTS (
	SELECT id FROM occupation
	WHERE id = user.occupation
);

DELETE FROM user
WHERE NOT EXISTS (
	SELECT zip FROM zipcode
	WHERE zip = user.zip
);

-- rating
DELETE FROM rating
WHERE NOT EXISTS (
	SELECT id FROM user
	WHERE id = rating.userId
);

DELETE FROM rating
WHERE NOT EXISTS (
	SELECT id FROM movie
	WHERE id = rating.movieId
);

-- moviegenre
DELETE FROM moviegenre
WHERE NOT EXISTS (
	SELECT id FROM genre
	WHERE id = moviegenre.genreId
);

DELETE FROM moviegenre
WHERE NOT EXISTS (
	SELECT id FROM movie
	WHERE id = moviegenre.movieId
);

-- zipcodedata
DELETE FROM zipcodedata
WHERE NOT EXISTS (
	SELECT zip FROM zipcode
	WHERE zip = zipcodedata.zip
);
%End SQL listing
\\
Age-value improvements: \\
%Begin SQL listing
-- age group (10-17)
UPDATE user
SET age=10+FLOOR(RAND()*8)
WHERE age=1;

-- age group (18-24)
UPDATE user
SET age=18+FLOOR(RAND()*7)
WHERE age=18;

-- age group (25-34)
UPDATE user
SET age=25+FLOOR(RAND()*10)
WHERE age=25;

-- age group (35-44)
UPDATE user
SET age=35+FLOOR(RAND()*10)
WHERE age=35;

-- age group (45-49)
UPDATE user
SET age=45+FLOOR(RAND()*5)
WHERE age=45;

-- age group (50-55)
UPDATE user
SET age=50+FLOOR(RAND()*6)
WHERE age=50;

-- age group (56-85)
UPDATE user
SET age=56+FLOOR(RAND()*30)
WHERE age=56;
%End SQL listing

\textbf{Enriching:} \\
For enriching, we had to first alter the .csv file in a way that we could read it properly. This involved removing the header at first, as we preferred making our own table befitting the data, and then loading the data into this table of the database, meaning the header data was irrelevant. The provided database used CHAR for its zip-values (so that a zip is allowed to start with 0) ahead of our enrichment, and because our .csv file contained zip codes of length 3 and 4, we had a choice to make; either alter the provided database or alter the data. We chose the latter. We thus went ahead and cycled through all 'zip' values in the dataset, adding 0s in front of all zip code entries with a length below 5, so that all values would have a length of 5, thus compatible with the provided database. The alternative would have been to alter the table in the provided database to hold a VARCHAR reference instead of a CHAR one. \\
Once the .csv file was edited, we had the task of creating a table to contain the data. We encountered the fact that for the exisiting table to be able to reference our new zipcodedata table, the two values (its foreign key column and zipcodedata's primary key column) had to have the exact same typing, including their character set definition. We found the correct character set and defined our new table to follow the norm of the rest of the database. \\
In the end, all there was left to do, was to load the file into our newly created table, lines separated by semi-colons. \\
See the relevant SQL queries for the enrichment below: \\

%Begin SQL listing
CREATE TABLE zipcodedata (
	zip char(5) CHARACTER SET latin1 COLLATE latin1_swedish_ci,
	mean int(11),
	pop int(11),
	PRIMARY KEY(zip)
);

LOAD DATA LOCAL INFILE 'C:/Users/malthe/Documents/GitHub/BIDD-Gruppe/Assignments/Assignment 4/MeanZIP-3.csv'
INTO TABLE zipcodedata
FIELDS TERMINATED BY ';';
%End SQL listing