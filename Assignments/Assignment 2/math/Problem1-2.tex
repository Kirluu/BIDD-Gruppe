(\rho_{[cpr \mapsto id]}R_1) \times R_2 = (\rho_{[cpr \mapsto id]}R_1) \times (\rho_{[cpr \mapsto id]}R_2)\\

\begin{table}[h]
\begin{tabular}{ll}
 R_{LHS} & R_{RHS}  \\
 (\rho_{[cpr \mapsto id]}R_1) \times R_2 & (\rho_{[cpr \mapsto id]R_1}) \times (\rho_{[cpr \mapsto id]R_2})  \\
 A(R_{LHS}) = (\rho_{[cpr\mapsto id]}(A(R_1)) \times A(R_2) & A(R_{RHS}) = (\rho_{[cpr\mapsto id]}A(R_1)) \times (\rho_{[cpr \mapsto id]}A(R_2))  \\
 A(R_{LHS}) = \{id, address\} \times \{cpr, name\} & A(R_{RHS}) = \{id, address\} \times \{id, name\}  \\
 A(R_{LHS}) = \{id, address, cpr, name\} & A(R_{RHS}) = \{id, address, id, name\}(undefined)
\end{tabular}
\end{table}

The equality is undefined as the righthand side relation R_{RHS} doesn't abide with the well-defined rule of cartesian product, which is that the relations must not have any common attributes. Even though a table like the above is created in MySQL when you try to select \em{id}\em it won't know which \em{id}\id attribute you are referring to.
For this equality to be true the RHS should not rename the cpr in R_2.