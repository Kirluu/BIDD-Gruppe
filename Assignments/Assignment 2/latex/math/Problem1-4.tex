\subsection{$\pi_{id,name,address}(\sigma_{cpr=id}(\rho_{[cpr \mapsto id]}R_1 \times R_2)) = (\rho_{[cpr \mapsto id]}R_1) \Join (\rho_{[cpr \mapsto id]}R_2)$}

\begin{table}[h]
	\begin{tabular}{ll}
		 $\bm{R_{LHS}}$ & $\bm{R_{RHS}}$  \\
		 $\pi_{id,name,address}(\sigma_{cpr=id}(\rho_{[cpr \mapsto id]}R_1 \times R_2))$ & $(\rho_{[cpr \mapsto id]}R_1) \Join (\rho_{[cpr \mapsto id]}R_2)$  \\
		 $A(R_{LHS}) = \pi_{id,name,address}(\sigma_{cpr=id}(\rho_{[cpr \mapsto id]}A(R_1) \times A(R_2)))$ & $A(R_{RHS}) = (\rho_{[cpr \mapsto id]}A(R_1) \Join (\rho_{[cpr \mapsto id]}A(R_2))$  \\
		 $A(R_{LHS}) = \pi_{id,name,address}(\sigma_{cpr=id}(\{id, address, cpr, name\} \times \{id, name\}))$ & $A(R_{RHS}) = \{id, address\} \Join \{id, name\}$  \\
		 $A(R_{LHS}) = \{id, address, name\}$ & $A(R_{RHS}) = \{id, address, name\}$
	\end{tabular}
\end{table}
\Floatbarrier

This is true as both the left-hand side relation and the right-hand side relation have the same attributes and tuples.