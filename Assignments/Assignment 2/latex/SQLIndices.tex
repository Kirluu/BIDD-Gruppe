\lstset{
	basicstyle=\normalsize
}

\section{Indices}
\begin{lstlisting}[language=sql]
CREATE INDEX nameIndex ON Person (name(10));
CREATE INDEX titleIndex ON Production (title(10));
CREATE INDEX genreNameIndex ON GenreType (genreName);
CREATE INDEX roleNameIndex ON Role (roleName);
\end{lstlisting}

\begin{figure}[ht]
	\centering
	\begin{tabular}{c|c|c}
		\textbf{Query} & \textbf{Before indices} & \textbf{After indices}\\
		\hline
		1 & 0.02 sec & 0.02 sec\\
		2 & 0.00 sec & 0.00 sec\\
		3 & 3.80 sec & 0.00 sec\\
		4 & 0.44 sec & 0.21 sec\\
		5 & 3.28 sec & 0.00 sec\\
		6 & 0.12 sec & 0.00 sec\\
		7 & 0.00 sec & 0.00 sec\\
		8 & 0.03 sec & 0.03 sec\\
		9 & 0.00 sec & 0.00 sec\\
		10 & 0.02 sec & 0.02 sec\\
		11 & 4.98 sec & 0.01 sec\\
		12 & 0.05 sec & 0.00 sec\\
		13 & 0.17 sec & 0.06 sec\\
		14 & 0.28 sec & 0.19 sec\\
		15 & 0.00 sec & 0.00 sec\\
		16 & 25.54 sec & 25.36 sec\\
		17 & 3.63 sec & 2.80 sec\\
		18 & 0.58 sec & 0.50 sec
	\end{tabular}
	\caption{Query time consumption with and without indices.}
\end{figure}

\subsection{Discussion}
We have chosen to create indices based on the attributes that we saw being requested most throughout the 18 queries that were necessary to support at the least. Performance improvements were seen upon all additions of indices. In the case of the index for roleName, improvements were actually seen in only one of three relevant queries. This corresponds to the EXPLAIN results (possible and actual indices used), as can be seen below, as the roleName index was considered by two queries, but only actually used in a single query.\\

Note that there is a possibility of the created indices being preferred where not appropriate. This could result in worsened performance for one of the required queries, as a mis-pick here would ultimately result in a suboptimal query plan.\\

Query nr. 16 is an unfortunate case, as the query written does not involve any attributes other than primary keys, and thus no possible indices that can be created would make any difference.\\

It can be observed that it is not in all cases that the indices we created are being used by the queries, despite it indeed being a possibility.

\begin{figure}[ht]
	\centering
	\begin{tabular}{c|c|c}
		\textbf{Index} & \textbf{Used key in query} & \textbf{Possible key in query}\\
		\hline
		\textbf{nameIndex}
		&
			\parbox{3cm}{
				\begin{itemize}
					\item Query 3
					\item Query 4
					\item Query 6
					\item Query 12
					\item Query 13
				\end{itemize}
			}
		&
			\parbox{3cm}{
				\begin{itemize}
					\item Query 17
				\end{itemize}
			}
		\\\hline
		\textbf{titleIndex}
		&
			\parbox{3cm}{
				\begin{itemize}
					\item Query 5
					\item Query 11
					\item Query 18
				\end{itemize}
			}
		&
			\parbox{3cm}{
				\begin{itemize}
					\item Query 15
				\end{itemize}
			}
		\\\hline
		\textbf{genreNameIndex}
		&
			\parbox{3cm}{
				\begin{itemize}
					\item Query 14
					\item Query 18
				\end{itemize}
			}
		&
		\\\hline
		\textbf{roleNameIndex}
		&
			\parbox{3cm}{
				\begin{itemize}
					\item Query 17
				\end{itemize}
			}
		&
			\parbox{3cm}{
				\begin{itemize}
					\item Query 4
					\item Query 11
				\end{itemize}
			}
	\end{tabular}
	\caption{Indices used as keys in queries}
\end{figure}