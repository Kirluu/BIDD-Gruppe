\section*{Description}
Below is the fact table that we have come up with. See below for the full explanation of its contents.

\begin{tabular}{ c | c | c | c | c | c }
  movieId & genreId & userId & occupationId & zip \\
  \hline
  1 & 42 & 3 & 2 & 06950 \\
  1 & 42 & 4 & 2 & 06950 \\
  1 & 42 & 1 & 2 & 06950 \\
  \hline  
\end{tabular} \\

\noindent One fact in this fact table represents (possibly only a part of) a single rating. Thus by using this fact table, we can say something about ratings on various movies which have different genres, as well as saying something about the users that made these ratings, specifically regarding age of user, demography and the associated income. \\
One fact in the fact table represents a single rating because one fact contains both a movieId and a userId, whereas these two combined grant access to rating data, since they form the primary key of the 'rating' table. \\
As it can be seen in the sample data above, one rating may occur several times. This is because one fact links to a single genreId, which means, in the case that a movie should have more than a single genre connected to it, it will occur several times as several facts - all these facts describe the same rating of the same movie concerning the same user. \\
The 'zip' value in the fact table links to two different tables; 'zipcode' and 'zipcodedata'. This is a case of vertical partitioning, as we let the zip value refer to two different data sources, all related to our user. \\

\noindent We have the following dimensions and possible measures by using this fact table: \\
Dimensions:

\begin{itemize}
  \item modelId
  \item userId
  \item genreId
  \item occupationId
  \item zip
\end{itemize}

\noindent Measures, deriviated from our dimensions:

\begin{itemize}
  \item rating (retrievable from movieId, userId (stored in rating table))
  \item time (date retrievable from movieId, userId (stored in rating table))
  \item lattitude (retrievable from zip)
  \item longitude (retrievable from zip)
  \item salary (retrievable from zip)
  \item population (retrieve from zip)
  \item age (retrievable from userId)
\end{itemize}

Found below are the transcripts for the fact table and the pre-aggregation tables along with the created indices. \\

\noindent\textbf{Fact table} \\
\begin{lstlisting}
CREATE TABLE ratingfacts (
  movieId INT(11),
  userId INT(11),
  genreId INT(11),
  occupationId INT(11),
  zip CHAR(5) CHARACTER SET latin1 COLLATE latin1_swedish_ci,
  FOREIGN KEY (movieId) REFERENCES movie(id),
  FOREIGN KEY (userId) REFERENCES user(id),
  FOREIGN KEY (genreId) REFERENCES genre(id),
  FOREIGN KEY (occupationId) REFERENCES occupation(id),
  FOREIGN KEY (zip) REFERENCES zipcode(zip)
);

INSERT INTO ratingfacts
SELECT rating.movieId, rating.userId, moviegenre.genreId, 
  occupation.id as occupation, zipcode.zip
FROM rating
JOIN user ON user.id = rating.userId
JOIN moviegenre ON moviegenre.movieId = rating.movieId
JOIN occupation ON occupation.id = user.occupation
JOIN zipcode ON zipcode.zip = user.zip;
\end{lstlisting}

\noindent\textbf{Pre-aggregation table: genreview} \\
\begin{lstlisting}
CREATE TABLE genreview (
  genreName VARCHAR(20),
  movieCount INT(11),
  ratingCount INT(11),
  ratingMean FLOAT,
  userSalaryMean INT(11),
  userAgeMean FLOAT,
  userLattitudeMean FLOAT,
  userLongitudeMean FLOAT,
  zipPopMean INT(11)
);

INSERT INTO genreview (
  SELECT
    genre.name,
    COUNT(DISTINCT facts.movieId),
    COUNT(DISTINCT facts.movieId, facts.userId),
    AVG(rating.rating),
    AVG(zipcodedata.mean),
    AVG(user.age),
    AVG(zipcode.lattitude),
    AVG(zipcode.longitude),
    AVG(zipcodedata.pop)
  FROM ratingfacts AS facts
  JOIN genre ON genre.id = facts.genreId
  JOIN rating ON rating.userId = facts.userId 
    AND rating.movieId = facts.movieId
  JOIN zipcode ON zipcode.zip = facts.zip
  JOIN zipcodedata ON zipcodedata.zip = facts.zip
  JOIN user ON user.id = facts.userId
  GROUP BY genreId
);

CREATE INDEX genreNameIndex ON genreview (genreName);
CREATE INDEX movieCountIndex ON genreview (movieCount);
CREATE INDEX ratingCountIndex ON genreview (ratingCount);
CREATE INDEX ratingMeanIndex ON genreview (ratingMean);
CREATE INDEX userSalaryMeanIndex ON genreview (userSalaryMean);
CREATE INDEX userAgeMeanIndex ON genreview (userAgeMean);
CREATE INDEX userLattitudeMeanIndex ON genreview (userLattitudeMean);
CREATE INDEX userLongitudeMeanIndex ON genreview (userLongitudeMean);
CREATE INDEX zipPopMeanIndex ON genreview (zipPopMean);
\end{lstlisting}

\noindent\textbf{Pre-aggregation table: zipcodeview}
\begin{lstlisting}
CREATE TABLE zipcodeview (
  zip INT(11),
  city VARCHAR(64),
  state CHAR(2),
  population INT(11),
  totalUsers INT(11),
  userAgeMean FLOAT,
  userLattitudeMean FLOAT,
  userLongitudeMean FLOAT,
  totalRatings INT(11),
  ratingMean FLOAT
);

INSERT INTO zipcodeview (
  SELECT
    facts.zip,
    zipcode.city,
    zipcode.state,
    zipcodedata.pop,
    COUNT(DISTINCT user.id),
    AVG(user.age),
    AVG(zipcode.lattitude),
    AVG(zipcode.longitude),
    COUNT(DISTINCT rating.userId, rating.movieId),
    AVG(rating.rating)
  FROM ratingfacts AS facts
  JOIN zipcode ON zipcode.zip = facts.zip
  JOIN zipcodedata ON zipcodedata.zip = facts.zip
  JOIN user ON user.id = facts.userId
  JOIN rating ON rating.userId = facts.userId 
    AND rating.movieId = facts.movieId
  GROUP BY zipcode.zip
);

CREATE INDEX zipIndex ON zipcodeview (zip);
CREATE INDEX cityIndex ON zipcodeview (city);
CREATE INDEX stateIndex ON zipcodeview (state);
CREATE INDEX populationIndex ON zipcodeview (population);
CREATE INDEX totalUsersIndex ON zipcodeview (totalUsers);
CREATE INDEX userAgeMeanIndex ON zipcodeview (userAgeMean);
CREATE INDEX userLattitudeMeanIndex ON zipcodeview (userLattitudeMean);
CREATE INDEX userLongitudeMeanIndex ON zipcodeview (userLongitudeMean);
CREATE INDEX totalRatingsIndex ON zipcodeview (totalRatings);
CREATE INDEX ratingMeanIndex ON zipcodeview (ratingMean);
\end{lstlisting}

\noindent\textbf{Pre-aggregation table: userview}
\begin{lstlisting}
CREATE TABLE userview (
  id INT(11),
  age FLOAT,
  occupation VARCHAR(20),
  lattitudeMean FLOAT,
  longitudeMean FLOAT,
  zip INT(11),
  city VARCHAR(64),
  state CHAR(2),
  population INT(11),
  totalRatings INT(11),
  ratingMean FLOAT
);

INSERT INTO userview (
  SELECT
    user.id,
    user.age,
    occupation.description,
    zipcode.lattitude,
    zipcode.longitude,
    zipcode.zip,
    zipcode.city,
    zipcode.state,
    zipcodedata.pop,
    COUNT(DISTINCT rating.userId, rating.movieId),
    AVG(rating.rating)
  FROM ratingfacts AS facts
  JOIN user ON user.id = facts.userId
  JOIN occupation ON occupation.id = user.occupation
  JOIN zipcode ON zipcode.zip = user.zip
  JOIN zipcodedata ON zipcodedata.zip = user.zip
  JOIN rating ON rating.userId = facts.userId 
    AND rating.movieId = facts.movieId
  JOIN moviegenre ON moviegenre.movieId = rating.movieId
  GROUP BY facts.userId
);

CREATE INDEX idIndex ON userview (id);
CREATE INDEX ageIndex ON userview (age);
CREATE INDEX occupationIndex ON userview (occupation);
CREATE INDEX lattitudeMeanIndex ON userview (lattitudeMean);
CREATE INDEX longitudeMeanIndex ON userview (longitudeMean);
CREATE INDEX zipIndex ON userview (zip);
CREATE INDEX cityIndex ON userview (city);
CREATE INDEX stateIndex ON userview (state);
CREATE INDEX populationIndex ON userview (population);
CREATE INDEX totalRatingsIndex ON userview (totalRatings);
CREATE INDEX ratingMeanIndex ON userview (ratingMean);
\end{lstlisting}