\section*{Data cleaning}

\noindent During data cleaning, we found that there was indeed references pointing from the user table toward zipcode tuples that did not exist. We thus deleted these users, to avoid any errors that may occur in the future. \\
After this, we moved on to look at the rating table, making sure that the ratings of the users we just deleted were also cleaned. Finally, we went ahead and checked the moviegenre table for invalid references, however there appeared to be none. \\
\indent Following this, we went ahead and applied proper structure to the database, by adding constraints that were otherwise not present - this was not mentioned in the assignment, but made our work a tad easier when executing various queries. \\
On the subject of the age data, we initially took the minimum age, and added it to a random value which was multiplied with the interval's length, to give some value in between. It appeared, however, that we had very few cases of the minimum age within each age group when we computed these ages. This was a problem, so we altered our update queries to 'FLOOR' the result, so that a RAND() computed value of 0.999, multiplied by 10, would, instead of becoming 10, become 9. This also meant that we could write the age intervals as their minimum up to their maximum allowed value minus one, since we can count on the flooring functionality to give us the results we want, as long as we define our numbers correctly. \\
\indent On a side-note, we decided to make the first (youngest users) interval go from the age of 10 to 18, instead of 0-18, to make our estimates of age more realistic.
See the relevant SQL queries for data cleaning below \\

\noindent\textbf{Primary keys} \\

\begin{lstlisting}
-- genre
ALTER TABLE genre
ADD PRIMARY KEY (id);

-- movie
ALTER TABLE movie
ADD PRIMARY KEY (id);

-- moviegenre
ALTER TABLE moviegenre
ADD PRIMARY KEY (genreId, movieId);

-- occupation
ALTER TABLE occupation
ADD PRIMARY KEY (id);

-- rating
ALTER TABLE rating
ADD PRIMARY KEY (userId, movieId);

-- user
ALTER TABLE user
ADD PRIMARY KEY (id);

-- zipcode
ALTER TABLE zipcode
ADD PRIMARY KEY (zip);
\end{lstlisting}
\bigskip

\noindent\textbf{Deletion-cleaning} \\

\begin{lstlisting}
-- user
DELETE FROM user
WHERE NOT EXISTS (
	SELECT id FROM occupation
	WHERE id = user.occupation
);

DELETE FROM user
WHERE NOT EXISTS (
	SELECT zip FROM zipcode
	WHERE zip = user.zip
);

-- rating
DELETE FROM rating
WHERE NOT EXISTS (
	SELECT id FROM user
	WHERE id = rating.userId
);

DELETE FROM rating
WHERE NOT EXISTS (
	SELECT id FROM movie
	WHERE id = rating.movieId
);

-- moviegenre
DELETE FROM moviegenre
WHERE NOT EXISTS (
	SELECT id FROM genre
	WHERE id = moviegenre.genreId
);

DELETE FROM moviegenre
WHERE NOT EXISTS (
	SELECT id FROM movie
	WHERE id = moviegenre.movieId
);

-- zipcodedata
DELETE FROM zipcodedata
WHERE NOT EXISTS (
	SELECT zip FROM zipcode
	WHERE zip = zipcodedata.zip
);
\end{lstlisting}
\bigskip

\noindent\textbf{Foreign keys} \\

\begin{lstlisting}
-- moviegenre
ALTER TABLE moviegenre
ADD FOREIGN KEY (genreId)
REFERENCES genre(id);

ALTER TABLE moviegenre
ADD FOREIGN KEY (movieId)
REFERENCES movie(id);

-- rating
ALTER TABLE rating
ADD FOREIGN KEY (userId)
REFERENCES user(id);

ALTER TABLE rating
ADD FOREIGN KEY (movieId)
REFERENCES movie(id);

-- user
ALTER TABLE user
ADD FOREIGN KEY (occupation)
REFERENCES occupation(id);

ALTER TABLE user
ADD FOREIGN KEY (zip)
REFERENCES zipcode(zip);
\end{lstlisting}

\noindent\textbf{Additional constraints} \\

\begin{lstlisting}
-- genre
DELETE FROM genre
WHERE name IS NULL;

ALTER TABLE genre
MODIFY name VARCHAR(20) NOT NULL;

-- movie
DELETE FROM movie
WHERE title IS NULL;

ALTER TABLE movie
MODIFY title VARCHAR(127) NOT NULL;

-- occupation
DELETE FROM occupation
WHERE description IS NULL;

ALTER TABLE occupation
MODIFY description VARCHAR(20) NOT NULL;

-- rating
DELETE FROM rating
WHERE rating IS NULL;

ALTER TABLE rating
MODIFY rating INT(11) NOT NULL;
\end{lstlisting}

\noindent\textbf{Age-value improvements} \\

\begin{lstlisting}
-- age group (10-17)
UPDATE user
SET age=10+FLOOR(RAND()*8)
WHERE age=1;

-- age group (18-24)
UPDATE user
SET age=18+FLOOR(RAND()*7)
,WHERE age=18;

-- age group (25-34)
UPDATE user
SET age=25+FLOOR(RAND()*10)
WHERE age=25;

-- age group (35-44)
UPDATE user
SET age=35+FLOOR(RAND()*10)
WHERE age=35;

-- age group (45-49)
UPDATE user
SET age=45+FLOOR(RAND()*5)
WHERE age=45;

-- age group (50-55)
UPDATE user
SET age=50+FLOOR(RAND()*6)
WHERE age=50;

-- age group (56-85)
UPDATE user
SET age=56+FLOOR(RAND()*30)
WHERE age=56;
\end{lstlisting}